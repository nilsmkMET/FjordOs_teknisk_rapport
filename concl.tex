\newpage
\section{Summary and final remarks}
\label{sec:summa}
A documentation of a new Oslofjord model is provided. The new Oslofjord model, named FjordOs CL, is based on the publically available ocean model ROMS \citep{shche:mcwil:2005,shche:mcwil:2009,haidv:etal:2008}, and is developed as part of the project FjordOs. FjordOs CL exploits the curvilinear option in ROMS to minimize the number of ``dry'' grid points at the expense of increasing the number of ``wet'' grid points. Thereby the grid resolution is enhanced and varies in space. In fact, the FjordOs CL mesh size varies from about 50 m in the {\DR} area to about 300 m at its southern border. 

To satisfy ourselves that the model works technically, is viable and produces results that are in line with our knowledge of the circulation in the fjord, we have run several test cases. Above we have shown examples of results from a hindcast case initiated on April 1, 2014 and run through December 31, 2015. A through validation of the results from this hindcast will be reported in a separate report.

In summary the results shown provides insight into the necessity of resolving the Oslofjord's irregular coastline geography, that is, the fjord's many small islands, narrow sounds and straits, and its topography, that is, deep basins and shallow areas. Thus, the new model provides a basis for developing an operation Oslofjord model once well identified and validated.
